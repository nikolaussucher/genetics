\hypertarget{molecular-genetics}{%
\chapter*{Molecular Genetics}\label{molecular-genetics}}
\addcontentsline{toc}{chapter}{Molecular Genetics}

In the remaining 8 laboratory sessions of this course, we will clone a
portion of a glyceraldehyde-3-phosphate dehydrogenase (GAPDH) gene from
plants, insert this gene fragment into a plasmid vector, and analyze the
sequences of resulting clones using bioinformatics.

Specifically, we will:

\begin{enumerate}
\def\labelenumi{\arabic{enumi}.}
\tightlist
\item
  Learn and practice some essential skills for the molecular laboratory
  (Chapter @ref(molecular-laboratory-skills))
\item
  Identify the plant or plants to be studied, extract the gDNA, and
  amplify a portion of the GAPC gene using PCR (Chapter
  @ref(dna-extraction-and-polymerase-chain-reaction)).
\item
  Assess the results of PCR, purify the PCR products and perform nested
  PCR (Chapter @ref(exonuclease-i-digestion-and-nested-pcr)).
\item
  Ligate (insert) the GAPC gene fragment into a plasmid vector and
  transform bacteria with the plasmid (Chapter
  @ref(agarose-gel-electrophoresis-ligation-and-transformation)).
\item
  Isolate the plasmid from the bacteria and confirm the presence of the
  insert by restriction enzyme digestion (Chapter
  @ref(plasmid-purification-and-restriction-digest)).
\item
  Prepare plasmid for DNA sequencing and send to a facility for
  sequencing (Chapter @ref(dna-sequencing)).
\item
  Obtain the sequence of the cloned GAPC gene fragment and analyze the
  cloned gene using bioinformatics (Chapter @ref(bioinformatics)).
\end{enumerate}

While we will chose and provide the plants, the reagents used in the
experiments will come in the form of ready made kits. In the early days
of molecular biology, researchers had to prepare all required materials
themselves including the laborious purification of enzymes. Today,
virtually any reagent that is required can be purchased
``off-the-shelf'', prepackaged and ready to use from a great number of
suppliers. One of the oldest companies specialized in supplying
materials for life science research laboratories are the
\href{https://en.wikipedia.org/wiki/Bio-Rad_Laboratories}{BioRad
Laboratories Inc.}, which developed the curriculum for these experiments
and supplies the materials under Bio-Rad Explorer\textsubscript{TM}
Cloning and Sequencing Explorer Series.

As our hands-on-time in the laboratory is limited, the RCC laboratory
technicians will help us by performing some behind the scenes work
between the lab sessions.

\href{https://en.wikipedia.org/wiki/Glyceraldehyde_3-phosphate_dehydrogenase}{Glyceraldehyde
3-phosphate dehydrogenase} (abbreviated as GAPDH or less commonly as
G3PDH) (EC 1.2.1.12) is an enzyme of \textasciitilde{}37kDa that
catalyzes the sixth step of glycolysis and thus serves to break down
glucose for energy and carbon molecules. In addition to this long
established metabolic function, GAPDH has recently been implicated in
several non-metabolic processes, including transcription activation,
initiation of apoptosis, ER to Golgi vesicle shuttling, and fast axonal,
or axoplasmic transport.

Because the GAPDH gene is often stably and constitutively expressed at
high levels in most tissues and cells, it is considered a housekeeping
gene. For this reason, GAPDH is commonly used by biological researchers
as a loading control for western blot and as a control for qPCR. Plants
have multiple GAPDH genes; the specific ones that have been selected for
cloning are the GAPC genes.
